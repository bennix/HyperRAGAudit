\chapter{文档处理与转换}
\label{ch:document-processing}

\section{问题定义}

企业审计中需要处理多种文档格式:PDF(最常见)、扫描件图片(PNG/JPG)、Word 文档(DOCX)。为了统一后续的 Vision LLM OCR 流程,第一步需要将所有格式转换为\textbf{标准化的页面图片}。

\begin{center}
\begin{tikzpicture}[
  node distance=1.5cm and 1cm,
  input/.style={rectangle, draw, fill=orange!10, minimum width=2cm, minimum height=0.8cm, font=\small},
  process/.style={rectangle, draw, rounded corners, fill=blue!10, minimum width=3cm, minimum height=0.8cm, font=\small},
  output/.style={rectangle, draw, fill=green!10, minimum width=2.5cm, minimum height=0.8cm, font=\small},
  arrow/.style={-{Stealth[length=2mm]}, thick}
]
  \node[input] (pdf) {PDF};
  \node[input, below=0.5cm of pdf] (img) {PNG / JPG};
  \node[input, below=0.5cm of img] (docx) {DOCX};
  \node[process, right=2cm of img] (conv) {DocConverter};
  \node[output, right=2cm of conv] (out) {List[PageImage]\\JPEG bytes};

  \draw[arrow] (pdf.east) -- (conv.west);
  \draw[arrow] (img.east) -- (conv.west);
  \draw[arrow] (docx.east) -- (conv.west);
  \draw[arrow] (conv) -- (out);
\end{tikzpicture}
\end{center}

\section{PageImage 数据结构}

每个页面转换后的结果封装为 \texttt{PageImage} 数据类:

\begin{lstlisting}[caption={PageImage 定义}]
@dataclass
class PageImage:
    page_num: int       # 页码(0-based)
    png_bytes: bytes    # 实际为 JPEG 字节(历史命名)
    width_px: int       # 图片宽度(像素)
    height_px: int      # 图片高度(像素)
\end{lstlisting}

\section{PDF 转换:PyMuPDF}

PDF 是审计文档的主要格式。使用 PyMuPDF\index{PyMuPDF}(即 \texttt{fitz})库将 PDF 的每一页渲染为图片。

\subsection{渲染流程}

\begin{lstlisting}[caption={PDF 页面渲染}]
import fitz  # PyMuPDF

def _convert_pdf(self, path: str) -> list[PageImage]:
    doc = fitz.open(path)
    pages = []
    for i, page in enumerate(doc):
        # DPI 转换矩阵:72 dpi -> 目标 dpi
        mat = fitz.Matrix(self._dpi / 72, self._dpi / 72)
        pix = page.get_pixmap(matrix=mat)

        # Pixmap -> JPEG bytes
        jpeg_bytes = self._to_jpeg_bytes(pix)

        pages.append(PageImage(
            page_num=i,
            png_bytes=jpeg_bytes,
            width_px=pix.width,
            height_px=pix.height,
        ))
    doc.close()
    return pages
\end{lstlisting}

\subsection{DPI 与图片质量}

\begin{notebox}[DPI 选择的权衡]
DPI(Dots Per Inch)决定了渲染精度与文件大小的平衡:
\begin{itemize}
  \item \textbf{72 dpi}:PDF 原始分辨率,文件小但文字可能模糊
  \item \textbf{150 dpi}(推荐):清晰度足够 OCR 识别,文件大小适中
  \item \textbf{300 dpi}:高清晰度,但文件体积是 150 dpi 的 4 倍
\end{itemize}
在实际测试中,300 dpi 的大图片会导致 API 传输超时(SSL 连接中断),因此 HyperRAG 默认使用 150 dpi。
\end{notebox}

\subsection{JPEG 压缩}

为了进一步减小传输体积,将 PyMuPDF 的 Pixmap(RGB 原始像素)转换为 JPEG 格式:

\begin{lstlisting}[caption={Pixmap 转 JPEG}]
from PIL import Image
import io

def _to_jpeg_bytes(self, pix) -> bytes:
    img = Image.frombytes("RGB", (pix.width, pix.height), pix.samples)
    buf = io.BytesIO()
    img.save(buf, format="JPEG", quality=self._jpeg_quality)
    return buf.getvalue()
\end{lstlisting}

\begin{table}[htbp]
\centering
\caption{不同压缩参数对 A4 页面的影响}
\label{tab:compression}
\begin{tabular}{lcc}
\toprule
\textbf{配置} & \textbf{文件大小} & \textbf{OCR 准确率} \\
\midrule
PNG @ 300 dpi & $\sim$2-5 MB & 最高 \\
JPEG 75 @ 150 dpi & $\sim$100-300 KB & 高 \\
JPEG 50 @ 100 dpi & $\sim$30-80 KB & 中等 \\
\bottomrule
\end{tabular}
\end{table}

\section{图片处理:PIL}

对于直接上传的图片文件(PNG/JPG),处理相对简单:

\begin{lstlisting}[caption={图片文件处理}]
def _convert_image(self, path: str) -> list[PageImage]:
    img = Image.open(path).convert("RGB")
    buf = io.BytesIO()
    img.save(buf, format="JPEG", quality=self._jpeg_quality)
    return [PageImage(
        page_num=0,
        png_bytes=buf.getvalue(),
        width_px=img.width,
        height_px=img.height,
    )]
\end{lstlisting}

图片文件只有一"页",因此返回包含单个 \texttt{PageImage} 的列表。

\section{Word 文档:python-docx}

DOCX 格式\index{DOCX}是结构化的 XML 文档,不需要 OCR 即可直接提取文本。python-docx 库提供了段落级别的访问:

\begin{lstlisting}[caption={DOCX 文本提取}]
from docx import Document

def _convert_docx(self, path: str) -> list[PageImage]:
    doc = Document(path)
    full_text = "\n".join(p.text for p in doc.paragraphs if p.text.strip())
    # 将文本渲染为图片(简化处理)
    # 实际实现中,DOCX 的文本直接参与索引,
    # 图片仅用于 UI 预览
    ...
\end{lstlisting}

\begin{warnbox}[DOCX 的局限性]
python-docx 只能提取文本段落,对于 DOCX 中嵌入的图片、复杂表格,提取能力有限。如果 DOCX 包含大量视觉元素,建议先将其导出为 PDF 再处理。
\end{warnbox}

\section{统一入口:convert 方法}

\texttt{DocConverter.convert()} 方法根据文件扩展名自动选择转换策略:

\begin{lstlisting}[caption={统一转换入口}]
def convert(self, file_path: str) -> list[PageImage]:
    ext = Path(file_path).suffix.lower()
    if ext == ".pdf":
        return self._convert_pdf(file_path)
    elif ext in (".png", ".jpg", ".jpeg"):
        return self._convert_image(file_path)
    elif ext == ".docx":
        return self._convert_docx(file_path)
    else:
        raise ValueError(f"Unsupported file type: {ext}")
\end{lstlisting}

\section{实践要点}

\begin{enumerate}
  \item \textbf{内存管理}:处理大 PDF(数百页)时,应逐页渲染而非一次性加载所有页面到内存中。HyperRAG 使用 \texttt{for page in doc} 迭代器模式。
  \item \textbf{错误处理}:损坏的 PDF 或不支持的加密格式会导致 PyMuPDF 抛出异常,应在 UI 层捕获并提示用户。
  \item \textbf{临时文件}:转换后的 JPEG 字节直接存在内存中(\texttt{PageImage.png\_bytes}),不写入磁盘,避免临时文件管理问题。
  \item \textbf{尺寸记录}:\texttt{width\_px} 和 \texttt{height\_px} 在后续 BBox 坐标转换中至关重要——Vision LLM 返回的坐标是相对于这个图片尺寸的。
\end{enumerate}
