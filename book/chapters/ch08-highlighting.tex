\chapter{PDF 溯源高亮}
\label{ch:highlighting}

\section{溯源的意义}

在审计工作中,"证据在哪里"与"发现了什么"同样重要。传统 RAG 系统只能返回文本片段,审计员还需要手动在原始文档中定位证据。HyperRAG 通过 PDF 高亮标注实现了\textbf{自动溯源}——点击"在 PDF 中查看",即可看到证据在原始文档上被高亮标出。

这个功能的实现依赖于一条关键的数据链路:

\begin{center}
\begin{tikzpicture}[
  node distance=0.5cm and 1cm,
  box/.style={rectangle, draw, rounded corners, fill=blue!8,
    minimum width=3.5cm, minimum height=0.8cm, font=\small, align=center},
  arrow/.style={-{Stealth[length=2mm]}, thick}
]
  \node[box] (v) {Vision LLM\\BBox [0-1000]};
  \node[box, right=1.5cm of v] (s) {ChromaDB\\元数据存储};
  \node[box, right=1.5cm of s] (a) {Agent\\AuditFinding};
  \node[box, below=1cm of a] (c) {坐标转换\\BBox $\rightarrow$ PDFRect};
  \node[box, left=1.5cm of c] (h) {PyMuPDF\\PDF 标注};

  \draw[arrow] (v) -- (s);
  \draw[arrow] (s) -- (a);
  \draw[arrow] (a) -- (c);
  \draw[arrow] (c) -- (h);
\end{tikzpicture}
\end{center}

\section{坐标系统}

系统中涉及三个不同的坐标系统:

\subsection{Vision LLM 归一化坐标}

Vision LLM 返回的 BBox 是归一化到 0-1000 的整数坐标:

\begin{itemize}
  \item 原点 $(0, 0)$:页面\textbf{左上角}
  \item 终点 $(1000, 1000)$:页面\textbf{右下角}
  \item 格式:$[\text{y\_min}, \text{x\_min}, \text{y\_max}, \text{x\_max}]$
\end{itemize}

\subsection{PDF 坐标系统}

PDF 使用\textbf{点}(Point, 1 pt = 1/72 inch)作为单位,但坐标原点在\textbf{左下角}:

\begin{itemize}
  \item 原点 $(0, 0)$:页面\textbf{左下角}
  \item $x$ 轴向右,$y$ 轴\textbf{向上}
  \item A4 纸张尺寸:$595.28 \times 841.89$ pt
\end{itemize}

\subsection{PyMuPDF 坐标系统}

PyMuPDF(fitz)使用的坐标系与 PDF 原始坐标\textbf{不同}:

\begin{itemize}
  \item 原点 $(0, 0)$:页面\textbf{左上角}
  \item $y$ 轴\textbf{向下}(与 PDF 原始坐标相反)
  \item 单位仍为点(pt)
\end{itemize}

\begin{notebox}[坐标系对比]
\begin{center}
\begin{tikzpicture}[scale=0.8]
  % Vision LLM
  \draw (0,0) rectangle (2.5,3.5);
  \fill[red] (0,3.5) circle (2pt);
  \node[above right, font=\small] at (0,3.5) {(0,0)};
  \node[below left, font=\small] at (2.5,0) {(1000,1000)};
  \draw[->] (0,3.5) -- (0,2.5) node[left,font=\tiny]{y};
  \draw[->] (0,3.5) -- (1,3.5) node[above,font=\tiny]{x};
  \node[below, font=\small\bfseries] at (1.25,-0.3) {Vision LLM};

  % PyMuPDF
  \begin{scope}[xshift=4cm]
  \draw (0,0) rectangle (2.5,3.5);
  \fill[blue] (0,3.5) circle (2pt);
  \node[above right, font=\small] at (0,3.5) {(0,0)};
  \node[below left, font=\small] at (2.5,0) {(W,H)};
  \draw[->] (0,3.5) -- (0,2.5) node[left,font=\tiny]{y};
  \draw[->] (0,3.5) -- (1,3.5) node[above,font=\tiny]{x};
  \node[below, font=\small\bfseries] at (1.25,-0.3) {PyMuPDF (pt)};
  \end{scope}

  % PDF native
  \begin{scope}[xshift=8cm]
  \draw (0,0) rectangle (2.5,3.5);
  \fill[green!50!black] (0,0) circle (2pt);
  \node[below right, font=\small] at (0,0) {(0,0)};
  \node[above left, font=\small] at (2.5,3.5) {(W,H)};
  \draw[->] (0,0) -- (0,1) node[left,font=\tiny]{y};
  \draw[->] (0,0) -- (1,0) node[below,font=\tiny]{x};
  \node[below, font=\small\bfseries] at (1.25,-0.3) {PDF 原生 (pt)};
  \end{scope}
\end{tikzpicture}
\end{center}
Vision LLM 和 PyMuPDF 的坐标原点相同(左上角),$y$ 轴方向也相同(向下),因此转换相对简单。
\end{notebox}

\section{坐标转换算法}

由于 Vision LLM 和 PyMuPDF 的坐标原点和 $y$ 轴方向一致,转换公式较为直接:

\begin{align}
  x_0^{\text{pdf}} &= \frac{x_\text{min}}{1000} \times W_{\text{page}} \label{eq:x0} \\
  y_0^{\text{pdf}} &= \frac{y_\text{min}}{1000} \times H_{\text{page}} \label{eq:y0} \\
  x_1^{\text{pdf}} &= \frac{x_\text{max}}{1000} \times W_{\text{page}} \label{eq:x1} \\
  y_1^{\text{pdf}} &= \frac{y_\text{max}}{1000} \times H_{\text{page}} \label{eq:y1}
\end{align}

其中 $W_{\text{page}}$ 和 $H_{\text{page}}$ 是 PDF 页面的宽度和高度(单位:点)。

\begin{lstlisting}[caption={坐标转换实现}]
import fitz

def gemini_bbox_to_pdf_rect(
    bbox: BBox,
    page: fitz.Page,
) -> fitz.Rect:
    """Convert normalised [0-1000] bbox to PDF Rect."""
    pw = page.rect.width
    ph = page.rect.height

    x0 = (bbox.x_min / 1000) * pw
    y0 = (bbox.y_min / 1000) * ph
    x1 = (bbox.x_max / 1000) * pw
    y1 = (bbox.y_max / 1000) * ph

    return fitz.Rect(x0, y0, x1, y1)
\end{lstlisting}

\section{PDF 高亮标注}

使用 PyMuPDF 的注释(Annotation)API 在 PDF 上添加高亮:

\begin{lstlisting}[caption={PDF 高亮实现}]
class PDFHighlighter:
    def highlight(
        self,
        pdf_path: str,
        locations: list[PDFRect],
        output_path: str,
        color: tuple = (1, 1, 0),    # 黄色
        opacity: float = 0.35,
    ) -> str:
        doc = fitz.open(pdf_path)

        for loc in locations:
            if loc.page_num >= len(doc):
                continue

            page = doc[loc.page_num]
            rect = fitz.Rect(loc.x0, loc.y0, loc.x1, loc.y1)

            # 添加高亮注释
            annot = page.add_highlight_annot(rect)
            annot.set_colors(stroke=color)
            annot.set_opacity(opacity)
            annot.update()

        doc.save(output_path)
        doc.close()
        return output_path
\end{lstlisting}

\subsection{高亮参数}

\begin{description}
  \item[color] RGB 三元组,范围 $[0, 1]$。默认 $(1, 1, 0)$ 为黄色。审计中也可使用红色 $(1, 0, 0)$ 标注高风险发现。
  \item[opacity] 透明度,范围 $[0, 1]$。0.35 是一个折中值——既能清晰看到高亮,又不遮挡原文。
\end{description}

\section{从 AuditFinding 到高亮 PDF}

完整的高亮流程:

\begin{enumerate}
  \item Agent 返回 \texttt{AuditFinding},其中 \texttt{source\_locations} 包含一组 \texttt{PDFRect}(已转换的 PDF 物理坐标)。
  \item UI 层调用 \texttt{PDFHighlighter.highlight()},传入原始 PDF 路径和坐标列表。
  \item PyMuPDF 在每个坐标位置添加高亮注释。
  \item 高亮后的 PDF 保存到 \texttt{data/highlighted/} 目录。
  \item UI 将高亮 PDF 以 base64 内嵌到 \texttt{<iframe>} 中显示。
\end{enumerate}

\begin{lstlisting}[caption={UI 层的高亮调用}]
def _show_highlighted_pdf(finding):
    for loc in finding.source_locations:
        for doc_id, path in st.session_state.file_paths.items():
            if path.lower().endswith(".pdf"):
                output = os.path.join(
                    settings.highlighted_dir,
                    f"highlighted_{finding.finding_id}.pdf",
                )
                highlighter.highlight(
                    pdf_path=path,
                    locations=finding.source_locations,
                    output_path=output,
                )
                # 显示高亮 PDF
                pdf_bytes = Path(output).read_bytes()
                b64 = base64.b64encode(pdf_bytes).decode()
                st.markdown(
                    f'<iframe src="data:application/pdf;'
                    f'base64,{b64}" width="100%" '
                    f'height="500"></iframe>',
                    unsafe_allow_html=True,
                )
\end{lstlisting}

\section{精度与误差}

BBox 坐标的精度受多个因素影响:

\begin{enumerate}
  \item \textbf{Vision LLM 的坐标精度}——不同模型的坐标精度不同,通常在 $\pm20$(相当于页面尺寸的 2\%)。
  \item \textbf{DPI 转换}——页面图片的 DPI 不影响归一化坐标,但影响 LLM 的"视觉分辨率"。
  \item \textbf{钳位截断}——超出 0-1000 范围的坐标被钳位到边界,可能导致高亮区域偏小。
\end{enumerate}

\begin{tipbox}[实践建议]
在实际使用中,建议对 BBox 做适当的\textbf{扩展}(如四边各扩展 10-20 个归一化单位),确保高亮区域完整覆盖目标内容,即使坐标有轻微偏差也不会遗漏。
\end{tipbox}
