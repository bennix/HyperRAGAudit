\chapter{Vision LLM 与结构化 OCR}
\label{ch:vision-ocr}

\section{传统 OCR 与 Vision LLM 的区别}

传统 OCR 引擎(如 Tesseract\index{Tesseract}、PaddleOCR\index{PaddleOCR})专注于字符识别,输出的是纯文本或带坐标的文字行。而 Vision LLM 的能力远超字符识别:

\begin{table}[htbp]
\centering
\caption{传统 OCR 与 Vision LLM 的对比}
\begin{tabular}{lp{5cm}p{5cm}}
\toprule
\textbf{维度} & \textbf{传统 OCR} & \textbf{Vision LLM} \\
\midrule
输出 & 文字行 + 字符坐标 & 结构化 JSON(文本、表格、图表描述) \\
表格处理 & 需要额外的表格检测模型 & 直接输出 Markdown 表格 \\
语义理解 & 无 & 能理解上下文、推断缺失信息 \\
多语言 & 需要分别配置 & 原生多语言支持 \\
成本 & 低(本地运行) & 较高(API 调用) \\
速度 & 快 & 较慢(网络延迟 + 推理时间) \\
\bottomrule
\end{tabular}
\end{table}

\section{提示词工程:OCR 系统提示}

Vision LLM 的输出质量很大程度上取决于提示词(Prompt)的设计。HyperRAG 的 OCR 系统提示词位于 \texttt{prompts/ocr\_system.txt}:

\begin{lstlisting}[language={},caption={OCR 系统提示词},basicstyle=\small\ttfamily]
You are a high-precision document digitisation assistant.

TASK: Parse the uploaded page image. Extract ALL text
content, tables, and figures.

REQUIREMENTS:
1. For every content block, provide bounding box coordinates
   as [y_min, x_min, y_max, x_max] normalised to a 0-1000
   scale relative to the full image dimensions.
2. Tables must be converted to Markdown format.
3. Figures/charts should be described textually.
4. Preserve the reading order of the document.
5. Output ONLY valid JSON -- no markdown fences, no commentary.

OUTPUT JSON SCHEMA:
{
  "content_blocks": [
    {
      "content_type": "text" | "table" | "figure",
      "text": "<extracted text or markdown table>",
      "bbox": [y_min, x_min, y_max, x_max]
    }
  ]
}

COORDINATE RULES:
- y_min, x_min = top-left corner of the bounding box
- y_max, x_max = bottom-right corner of the bounding box
- All values are integers in the range [0, 1000]
- (0, 0) is the top-left; (1000, 1000) is the bottom-right
\end{lstlisting}

\subsection{提示词设计要点}

\begin{enumerate}
  \item \textbf{明确输出格式}——指定 JSON Schema,减少模型"自由发挥"的空间。
  \item \textbf{坐标约定}——明确 \texttt{[y\_min, x\_min, y\_max, x\_max]} 的顺序和范围。不同模型可能有不同的默认坐标约定,必须在提示词中显式说明。
  \item \textbf{内容类型枚举}——限定 \texttt{text}、\texttt{table}、\texttt{figure} 三种类型,后续处理代码只需匹配这三种。
  \item \textbf{禁止额外输出}——"Output ONLY valid JSON"防止模型在 JSON 前后添加解释性文字。
\end{enumerate}

\section{Vision API 调用}

\subsection{图片编码}

将 JPEG 字节编码为 base64 字符串,嵌入 OpenAI Vision API 的消息格式:

\begin{lstlisting}[caption={Base64 图片编码与 API 调用}]
import base64

def _parse_page(self, page: PageImage) -> PageInfo:
    b64 = base64.b64encode(page.png_bytes).decode("utf-8")

    messages = [
        {"role": "system", "content": self._system_prompt},
        {"role": "user", "content": [
            {
                "type": "text",
                "text": "Extract all content with bounding boxes "
                        "from this document page.",
            },
            {
                "type": "image_url",
                "image_url": {
                    "url": f"data:image/jpeg;base64,{b64}",
                },
            },
        ]},
    ]

    resp = self._client.chat.completions.create(
        model=self._model,
        messages=messages,
        max_tokens=self._max_tokens,
        temperature=0.0,
    )
    raw = resp.choices[0].message.content
    return self._parse_response(raw, page)
\end{lstlisting}

\begin{notebox}[temperature = 0.0 的重要性]
对于 OCR 任务,我们希望输出尽可能确定性和一致性,因此将 \texttt{temperature} 设为 0.0。这意味着模型几乎总是选择概率最高的 token,减少随机性。
\end{notebox}

\subsection{MIME 类型}

\texttt{data:image/jpeg;base64,\{b64\}} 中的 MIME 类型必须与实际图片格式匹配。由于我们在 DocConverter 中将所有图片转换为 JPEG,这里统一使用 \texttt{image/jpeg}。

\section{响应解析}

Vision LLM 的响应并不总是完美的 JSON。实践中可能遇到以下情况:

\begin{enumerate}
  \item 模型在 JSON 外面包裹了 Markdown 代码围栏(\texttt{```json ... ```})
  \item JSON 前后有解释性文字
  \item BBox 坐标超出 0-1000 范围
  \item JSON 格式错误(罕见但可能发生)
\end{enumerate}

\subsection{JSON 提取}

\texttt{\_extract\_json} 方法处理前两种情况:

\begin{lstlisting}[caption={从 LLM 响应中提取 JSON}]
import re

@staticmethod
def _extract_json(text: str) -> str:
    """Strip markdown fences and surrounding text."""
    # 尝试匹配 ```json ... ``` 代码块
    m = re.search(r"```(?:json)?\s*\n?(.*?)```", text, re.DOTALL)
    if m:
        return m.group(1).strip()

    # 回退:找到第一个 { 和最后一个 }
    start = text.find("{")
    end = text.rfind("}")
    if start != -1 and end != -1 and end > start:
        return text[start : end + 1]

    return text.strip()
\end{lstlisting}

\subsection{坐标钳位(Clamping)}

Vision LLM 有时会返回超出 0-1000 范围的坐标(如 1040),这会导致 Pydantic 验证失败。解决方案是钳位函数:

\begin{lstlisting}[caption={坐标钳位}]
def _clamp(v: int) -> int:
    return max(0, min(1000, v))

bbox = BBox(
    y_min=_clamp(int(bbox_list[0])),
    x_min=_clamp(int(bbox_list[1])),
    y_max=_clamp(int(bbox_list[2])),
    x_max=_clamp(int(bbox_list[3])),
)
\end{lstlisting}

\begin{warnbox}[防御性编程]
与 LLM 交互时,永远不要假设输出格式是完美的。即使提示词要求 "only valid JSON",模型仍可能输出非预期格式。系统的每一层都应有容错机制。
\end{warnbox}

\subsection{内容类型映射}

LLM 返回的 \texttt{content\_type} 可能有意料之外的值(如 "paragraph"、"heading"),需要映射到已定义的枚举:

\begin{lstlisting}[caption={容错的内容类型映射}]
ct_raw = item.get("content_type", "text").lower()
try:
    ct = ContentType(ct_raw)
except ValueError:
    ct = ContentType.TEXT  # 未知类型回退为 text
\end{lstlisting}

\section{逐页解析与进度反馈}

\texttt{parse\_single\_page} 方法将解析逻辑封装为单页粒度,便于 UI 层展示实时进度:

\begin{lstlisting}[caption={逐页解析}]
def parse_single_page(self, page: PageImage,
                      page_idx: int, total: int) -> PageInfo:
    t0 = time.time()
    img_kb = len(page.png_bytes) / 1024
    log.info(
        f"[OCR] Page {page_idx + 1}/{total} | "
        f"size={page.width_px}x{page.height_px} | "
        f"image={img_kb:.0f}KB | sending to {self._model}..."
    )
    page_info = self._parse_page(page)
    elapsed = time.time() - t0
    log.info(
        f"[OCR] Page {page_idx + 1}/{total} done "
        f"in {elapsed:.1f}s | "
        f"{len(page_info.content_blocks)} content blocks"
    )
    return page_info
\end{lstlisting}

在 Streamlit 层使用 \texttt{st.progress} 结合此方法:

\begin{lstlisting}[caption={UI 进度条}]
progress = st.progress(0, text="OCR parsing...")
for i, page in enumerate(pages):
    progress.progress(i / total, text=f"Page {i+1}/{total}")
    page_info = parser.parse_single_page(page, i, total)
    parsed_pages.append(page_info)
progress.progress(1.0, text="OCR complete")
\end{lstlisting}

\section{性能优化策略}

\begin{enumerate}
  \item \textbf{降低 DPI}:从 300 降到 150,图片体积减少 75\%,API 传输更快。
  \item \textbf{JPEG 压缩}:quality=75 在清晰度和体积之间取得良好平衡。
  \item \textbf{超时与重试}:设置 \texttt{httpx.Timeout(300.0, connect=30.0)} 和 \texttt{max\_retries=3},应对网络波动。
  \item \textbf{模型选择}:不同 Vision 模型的速度差异巨大。可通过修改 \texttt{config.yaml} 灵活切换。
\end{enumerate}
