\chapter{Streamlit Web 界面}
\label{ch:web-ui}

\section{为什么选择 Streamlit}

构建 RAG 系统的 Web 界面有多种选择:

\begin{table}[htbp]
\centering
\caption{Web 框架对比}
\begin{tabular}{lp{4cm}p{4cm}}
\toprule
\textbf{框架} & \textbf{优势} & \textbf{劣势} \\
\midrule
Streamlit & 纯 Python,快速原型,内置组件丰富 & 定制性有限,不适合复杂交互 \\
Gradio & 专为 ML Demo 设计,简单易用 & 布局灵活性差 \\
Flask + React & 完全自定义,前后端分离 & 开发成本高 \\
FastAPI + Vue & 高性能 API,现代前端 & 技术栈复杂 \\
\bottomrule
\end{tabular}
\end{table}

对于 HyperRAG 这类数据分析导向的应用,Streamlit\index{Streamlit} 是最佳选择:

\begin{itemize}
  \item 100\% Python,无需前端开发经验
  \item 内置文件上传、进度条、表格、PDF 嵌入等组件
  \item \texttt{st.session\_state} 自动管理应用状态
  \item 修改代码后自动刷新,开发效率极高
\end{itemize}

\section{应用结构}

HyperRAG 的 Streamlit 应用采用\textbf{侧边栏 + 三 Tab} 的布局:

\begin{lstlisting}[caption={应用主框架}]
def main() -> None:
    st.set_page_config(
        page_title="HyperRAG Audit",
        layout="wide",
    )
    _init_session_state()

    with st.sidebar:
        render_sidebar()

    tab_docs, tab_audit, tab_graph = st.tabs(
        [t("tab_docs"), t("tab_audit"), t("tab_graph")]
    )

    with tab_docs:
        render_document_view()
    with tab_audit:
        render_audit_query()
    with tab_graph:
        render_knowledge_graph()
\end{lstlisting}

\begin{description}
  \item[侧边栏] 语言切换、文件上传、已上传文档列表、重建知识图谱按钮
  \item[Tab 1: 文档预览] 原始 PDF 与解析内容的双栏对比
  \item[Tab 2: 审计查询] 查询输入、Agent 推理过程、审计发现与 PDF 高亮
  \item[Tab 3: 知识图谱] 交互式图谱可视化、实体搜索与过滤
\end{description}

\section{Session State 管理}

Streamlit 的每次用户交互都会重新执行整个脚本。为了在交互之间保持状态,使用 \texttt{st.session\_state}:

\begin{lstlisting}[caption={Session State 初始化}]
def _init_session_state() -> None:
    if "initialised" in st.session_state:
        return  # 已初始化,跳过

    settings = Settings()

    # 创建所有服务实例
    factory = LLMClientFactory(
        api_key=settings.zenmux_api_key,
        base_url=settings.zenmux_base_url,
    )
    st.session_state.settings = settings
    st.session_state.factory = factory
    st.session_state.openai_client = factory.get_openai_client()
    st.session_state.claude_llm = factory.get_langchain_llm(
        model=settings.claude_model,
    )
    st.session_state.doc_converter = DocConverter(...)
    st.session_state.gemini_parser = GeminiParser(...)
    st.session_state.vector_store = VectorStore(...)
    st.session_state.kg_store = KnowledgeGraphStore()
    ...

    st.session_state.parsed_docs = {}   # doc_id -> ParsedDocument
    st.session_state.uploaded_files = {}  # filename -> doc_id
    st.session_state.file_paths = {}      # doc_id -> file path
    st.session_state.kg_built = False

    st.session_state.initialised = True
\end{lstlisting}

\begin{notebox}[初始化守卫模式]
\texttt{if "initialised" in st.session\_state: return} 这个模式确保了昂贵的初始化操作(如创建数据库连接、加载配置)只执行一次。每次用户点击按钮或切换 Tab 时,脚本重新执行但会跳过初始化。
\end{notebox}

\section{文件上传与处理流程}

上传流程是应用中最复杂的部分,涉及多个异步步骤和进度反馈:

\begin{lstlisting}[caption={上传处理流程}]
if uploaded and st.button(t("upload_btn"), type="primary"):
    for f in uploaded:
        # Step 1: 保存文件
        save_path = os.path.join(settings.upload_dir, f.name)
        with open(save_path, "wb") as fp:
            fp.write(f.getbuffer())

        # Step 2: 转换为页面图片
        with st.status(t("converting", name=f.name)):
            pages = doc_converter.convert(save_path)

        # Step 3: 逐页 OCR(带进度条)
        progress = st.progress(0, text="OCR...")
        parsed_pages = []
        for i, page in enumerate(pages):
            progress.progress(i / total,
                text=t("ocr_progress", cur=i+1, total=total))
            page_info = parser.parse_single_page(page, i, total)
            parsed_pages.append(page_info)
        progress.progress(1.0, text=t("ocr_done"))

        # Step 4: 向量索引
        n = vector_store.add_document(parsed)

        # Step 5: 知识图谱构建(带进度条)
        kg_progress = st.progress(0, text="KG...")
        for i, page in enumerate(parsed.pages):
            kg_progress.progress(i / total, ...)
            ents, rels = extractor._extract_from_page(...)
            ...
        kg_progress.progress(1.0)
\end{lstlisting}

\subsection{进度反馈设计}

HyperRAG 使用两种进度反馈机制:

\begin{enumerate}
  \item \textbf{st.status}——用于简短的同步操作(如文件转换),显示为可展开的状态卡片。
  \item \textbf{st.progress}——用于逐页处理的长操作(如 OCR 和 KG 构建),显示为进度条并实时更新文本。
\end{enumerate}

\section{国际化(i18n)}

HyperRAG 支持中英文切换,采用字典映射方案:

\begin{lstlisting}[caption={i18n 实现}]
I18N = {
    "en": {
        "page_title": "HyperRAG Audit",
        "upload_header": "Upload Documents",
        "upload_btn": "Upload & Parse",
        ...
    },
    "zh": {
        "page_title": "HyperRAG 智能审计",
        "upload_header": "上传文档",
        "upload_btn": "上传并解析",
        ...
    },
}

def t(key: str, **kwargs) -> str:
    """Get translated string for current language."""
    lang = st.session_state.get("lang", "zh")
    text = I18N.get(lang, I18N["en"]).get(key, key)
    if kwargs:
        text = text.format(**kwargs)
    return text
\end{lstlisting}

\texttt{t()} 函数支持参数插值:\texttt{t("ocr\_progress", cur=3, total=10)} 会渲染为 "OCR: 正在解析第 3/10 页..."。

\section{知识图谱可视化}

使用 \texttt{streamlit-agraph}\index{streamlit-agraph} 库实现交互式图谱:

\begin{lstlisting}[caption={知识图谱可视化}]
from streamlit_agraph import agraph, Node, Edge, Config

# 为每个实体类型定义颜色
type_colors = {
    "Person": "#2e7d32",
    "Company": "#1565c0",
    "Amount": "#e65100",
    ...
}

# 使用数值 ID 避免 CJK 字符的 URL 编码问题
name_to_id = {n["name"]: f"n{i}" for i, n in enumerate(nodes_data)}

nodes = [
    Node(
        id=name_to_id[n["name"]],
        label=n["name"][:20],  # 截断长标签
        size=25,
        color=type_colors.get(n.get("entity_type"), "#999"),
        title=f'{n.get("entity_type")}: {n["name"]}',
    )
    for n in nodes_data
]

edges = [
    Edge(
        source=name_to_id[e["source"]],
        target=name_to_id[e["target"]],
        label=e.get("relation_type", ""),
    )
    for e in edges_data
]

config = Config(
    width=900, height=500,
    directed=True, physics=True,
)
agraph(nodes=nodes, edges=edges, config=config)
\end{lstlisting}

\begin{warnbox}[CJK 字符的坑]
\texttt{streamlit-agraph} 内部将节点 ID 用于 URL 路径。如果 ID 包含中文字符,会导致 \texttt{FileNotFoundError}。解决方案是使用数值 ID(如 \texttt{n0}, \texttt{n1}),将中文名称放在 \texttt{label} 和 \texttt{title} 属性中。
\end{warnbox}

\section{PDF 内嵌显示}

Streamlit 没有原生的 PDF 查看器组件,使用 HTML \texttt{<iframe>} 内嵌 base64 编码的 PDF:

\begin{lstlisting}[caption={PDF 内嵌显示}]
pdf_bytes = Path(file_path).read_bytes()
b64 = base64.b64encode(pdf_bytes).decode()
pdf_html = (
    f'<iframe src="data:application/pdf;base64,{b64}" '
    f'width="100%" height="600" '
    f'type="application/pdf"></iframe>'
)
st.markdown(pdf_html, unsafe_allow_html=True)
\end{lstlisting}

\begin{tipbox}[unsafe\_allow\_html]
\texttt{unsafe\_allow\_html=True} 允许注入原始 HTML。仅在你完全控制 HTML 内容时使用。不要将用户输入直接拼入 HTML 字符串,防止 XSS 攻击。
\end{tipbox}
