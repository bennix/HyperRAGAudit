\chapter{LangGraph 审计 Agent}
\label{ch:agent}

\section{什么是 Agent}

在 LLM 应用中,Agent\index{Agent} 是一个能够\textbf{自主决定使用哪些工具、以何种顺序执行}的智能体。与简单的"用户提问 $\rightarrow$ 检索 $\rightarrow$ 生成回答"流程不同,Agent 可以:

\begin{enumerate}
  \item 分析用户问题,决定需要哪些信息
  \item 调用工具获取信息
  \item 分析工具返回的结果
  \item 决定是否需要更多信息(再次调用工具)
  \item 最终综合所有信息生成回答
\end{enumerate}

这种多步推理能力对审计场景至关重要——审计员不会只看一个文档片段就下结论,而是会交叉查证多个信息来源。

\section{ReAct 模式}

ReAct\index{ReAct}(Reasoning + Acting)是目前最流行的 Agent 架构模式。它让 LLM 在每一步交替进行\textbf{思考}(Reasoning)和\textbf{行动}(Acting):

\begin{center}
\begin{tikzpicture}[
  node distance=1cm,
  think/.style={rectangle, draw, rounded corners, fill=yellow!15,
    minimum width=4cm, minimum height=0.8cm, font=\small},
  act/.style={rectangle, draw, rounded corners, fill=blue!10,
    minimum width=4cm, minimum height=0.8cm, font=\small},
  obs/.style={rectangle, draw, rounded corners, fill=green!10,
    minimum width=4cm, minimum height=0.8cm, font=\small},
  arrow/.style={-{Stealth[length=2mm]}, thick}
]
  \node[think] (t1) {Thought: 需要查找合同金额};
  \node[act, below=of t1] (a1) {Action: search\_documents("合同金额")};
  \node[obs, below=of a1] (o1) {Observation: 找到 3 个相关片段...};
  \node[think, below=of o1] (t2) {Thought: 需要对比发票金额};
  \node[act, below=of t2] (a2) {Action: query\_knowledge\_graph("发票")};
  \node[obs, below=of a2] (o2) {Observation: 发票关联实体...};
  \node[think, below=of o2] (t3) {Thought: 发现金额不一致,生成报告};

  \draw[arrow] (t1) -- (a1);
  \draw[arrow] (a1) -- (o1);
  \draw[arrow] (o1) -- (t2);
  \draw[arrow] (t2) -- (a2);
  \draw[arrow] (a2) -- (o2);
  \draw[arrow] (o2) -- (t3);
\end{tikzpicture}
\end{center}

\section{LangGraph 框架}

LangGraph\index{LangGraph} 是 LangChain 团队开发的 Agent 框架,基于图(Graph)的概念组织 Agent 的执行流程。

\subsection{为什么选择 LangGraph}

\begin{itemize}
  \item \textbf{原生 ReAct 支持}——\texttt{create\_react\_agent} 一行代码即可创建完整的 ReAct Agent
  \item \textbf{流式输出}——原生支持流式返回中间步骤,适合 UI 实时显示推理过程
  \item \textbf{状态管理}——自动管理对话历史和工具调用状态
  \item \textbf{可扩展}——可以轻松添加自定义节点和边
\end{itemize}

\subsection{Agent 创建}

\begin{lstlisting}[caption={创建 ReAct Agent}]
from langgraph.prebuilt import create_react_agent

class AuditAgent:
    def __init__(self, llm, tools, prompts_dir, max_iterations):
        self._system_prompt = self._load_prompt(prompts_dir)
        self._max_iterations = max_iterations

        self._agent = create_react_agent(
            model=llm,
            tools=tools,
            prompt=self._system_prompt,
        )
\end{lstlisting}

\section{工具设计}

Agent 的能力由其可用工具决定。HyperRAG 为审计 Agent 设计了 5 个专用工具:

\subsection{工具清单}

\begin{table}[htbp]
\centering
\caption{Agent 工具清单}
\begin{tabular}{lp{8cm}}
\toprule
\textbf{工具名称} & \textbf{功能} \\
\midrule
\texttt{search\_documents} & 在向量数据库中进行语义搜索,返回最相关的文档片段及其 BBox \\
\texttt{query\_knowledge\_graph} & 查询知识图谱中某个实体的邻居节点和关系 \\
\texttt{query\_graph\_path} & 查询两个实体之间的连接路径 \\
\texttt{get\_page\_content} & 获取指定文档的指定页面的完整内容 \\
\texttt{list\_entities\_by\_type} & 列出知识图谱中某类型的所有实体 \\
\bottomrule
\end{tabular}
\end{table}

\subsection{工具实现示例}

使用 LangChain 的 \texttt{@tool} 装饰器定义工具:

\begin{lstlisting}[caption={search\_documents 工具}]
from langchain_core.tools import tool

def create_tools(vector_store, kg_store, parsed_docs):

    @tool
    def search_documents(query: str) -> str:
        """Search documents by semantic similarity.
        Returns relevant text chunks with source info."""
        results = vector_store.search(query, top_k=5)
        if not results:
            return "No relevant documents found."
        output = []
        for r in results:
            output.append(
                f"[Doc:{r.doc_id} Page:{r.page_num+1} "
                f"Score:{r.score:.2f}]\n{r.text[:500]}"
            )
        return "\n---\n".join(output)

    @tool
    def query_knowledge_graph(entity_name: str) -> str:
        """Query the knowledge graph for an entity's
        relationships and neighbors."""
        result = kg_store.query_neighbors(
            entity_name, depth=2
        )
        if not result["nodes"]:
            return f"Entity '{entity_name}' not found."
        return json.dumps(result, ensure_ascii=False,
                          indent=2)

    ...
    return [search_documents, query_knowledge_graph,
            query_graph_path, get_page_content,
            list_entities_by_type]
\end{lstlisting}

\begin{notebox}[工具 Docstring 的重要性]
LLM 通过工具的 \textbf{docstring}(函数文档字符串)来理解工具的用途和使用方式。一个好的 docstring 应该:
\begin{itemize}
  \item 清晰描述工具的功能
  \item 说明输入参数的含义
  \item 描述返回值的格式
\end{itemize}
LLM 会根据 docstring 决定何时使用哪个工具以及传入什么参数。
\end{notebox}

\subsection{工厂模式}

工具通过 \texttt{create\_tools} 工厂函数创建,接收依赖项作为参数。这种模式的优势:

\begin{enumerate}
  \item \textbf{依赖注入}——工具函数通过闭包访问 \texttt{vector\_store}、\texttt{kg\_store} 等对象,不依赖全局变量。
  \item \textbf{可测试性}——单元测试可以传入 mock 对象。
  \item \textbf{解耦}——工具定义与存储实现分离。
\end{enumerate}

\section{流式推理}

审计查询的推理过程可能持续数十秒,为了让用户了解 Agent 的工作进展,使用流式输出:

\begin{lstlisting}[caption={流式推理输出}]
def stream(self, query: str):
    """Yield reasoning steps as they happen."""
    inputs = {"messages": [("user", query)]}
    config = {"recursion_limit": self._max_iterations * 2}

    for event in self._agent.stream(inputs, config):
        for node_name, node_output in event.items():
            messages = node_output.get("messages", [])
            for msg in messages:
                if hasattr(msg, "tool_calls") and msg.tool_calls:
                    for tc in msg.tool_calls:
                        yield {
                            "type": "tool_call",
                            "content": f"{tc['name']}({tc['args']})"
                        }
                elif hasattr(msg, "content") and msg.content:
                    if msg.type == "tool":
                        yield {
                            "type": "tool_result",
                            "content": msg.content[:500]
                        }
                    else:
                        yield {
                            "type": "thinking",
                            "content": msg.content
                        }
\end{lstlisting}

在 Streamlit 中,每个 yield 的步骤会实时显示在界面上,用户可以看到 Agent 正在调用哪个工具、获取了什么结果。

\section{审计报告生成}

Agent 的最终输出是一个结构化的审计报告:

\begin{lstlisting}[language=json,caption={审计报告 JSON 结构}]
{
  "findings": [
    {
      "finding_id": "F001",
      "description": "合同金额与发票总额不一致",
      "severity": "high",
      "evidence": [
        "合同 HW-2024-001 金额为 500 万元",
        "对应发票总额为 520 万元"
      ],
      "source_pages": [
        {"doc_id": "abc123", "page_num": 3}
      ],
      "related_entities": ["HW-2024-001", "500万元"]
    }
  ],
  "summary": "发现 1 项高风险问题..."
}
\end{lstlisting}

\subsection{报告解析}

与 OCR 响应类似,Agent 输出的 JSON 也需要容错解析:

\begin{lstlisting}[caption={报告解析}]
def _parse_report(self, query: str, content: str):
    cleaned = self._extract_json(content)
    try:
        data = json.loads(cleaned)
    except json.JSONDecodeError:
        return AuditReport(
            query=query,
            findings=[],
            summary=content,  # 回退:将原文作为摘要
        )
    # 构建 AuditReport...
\end{lstlisting}

\section{Agent 系统提示词}

位于 \texttt{prompts/audit\_system.txt} 的系统提示词定义了 Agent 的角色和行为准则:

\begin{enumerate}
  \item \textbf{角色定义}——"You are an expert financial and compliance auditor AI assistant"
  \item \textbf{工具说明}——列出所有可用工具及其用途
  \item \textbf{工作流程}——理解 $\rightarrow$ 搜索 $\rightarrow$ 交叉验证 $\rightarrow$ 识别异常 $\rightarrow$ 生成报告
  \item \textbf{输出格式}——JSON Schema 定义
  \item \textbf{行为规则}——引用具体页码、分级严重程度、不捏造证据
\end{enumerate}

\begin{tipbox}[迭代次数限制]
\texttt{max\_iterations} 限制了 Agent 的最大推理步数(默认 10 步)。这是一个安全阀——防止 Agent 陷入无限循环。在实践中,大多数审计查询在 3-6 步内完成。
\end{tipbox}
